\documentclass[]{article}
\usepackage[portrait, margin=5mm]{geometry}
\usepackage{amsmath}
\usepackage{amssymb}
\usepackage{amsfonts}
\usepackage{listings}
\pagenumbering{gobble}

\begin{document}
	
\lstset{
	basicstyle=\ttfamily\small,
	literate=
	{→}{{$\rightarrow$}}1
	{∀}{{$\forall$}}1
	{∃}{{$\exists$}}1
	{×}{{$\times$}}1
	{σ}{{$\sigma$}}1
	{τ}{{$\tau$}}1
	{≠}{{$\neq$}}1
	{≤}{{$\leq$}}1
	{≥}{{$\geq$}}1
	{↔}{{$\iff$}}1
	{¬}{{$\neg$}}1
	{∧}{{$\wedge$}}1
	{•}{$\bullet$}1
	{·}{$\cdot$}1
	{⬝}{$\cdot$}1
	{ℕ}{{$\mathbb{N}$}}1
	{ₗ}{{$_l$}}1
	{₀}{{$_0$}}1
	{∑}{{$\sum$}}1
	{ᵀ}{{$^\texttt{T}$}}1
	{ᵥ}{{$_v$}}1
}

Farkas established that a system of linear inequalities has a solution if and only if we cannot obtain
a contradiction by taking a nonnegative linear combination of the inequalities.
We state and formally prove several Farkas-like theorems in Lean 4.
Furthermore, we consider the rational numbers extended with two special elements denoted by $\bot$ and $\top$
where $\bot$ is below every rational number and $\top$ is above every rational number.
We define $\bot + a = \bot = a + \bot$ of all $a$ and we define $\top + b = \top = b + \top$ for all $b \neq \bot$.
For multiplication, we define $\bot \cdot c = \bot = c \cdot \bot$ for every $c \ge 0$ but
$\top \cdot d = \top = d \cdot \top$ only for $d > 0$ because $\top \cdot 0 = 0 = 0 \cdot \top$.
In this setting, where $\bot$ acts as negative infinity and $\top$ acts as positive infinity, with
$\bot$ being stronger than $\top$ in the elementary arithmetical operations, we extend certain versions of
the Farkas-like theorems about matrices to a setting where coefficients are extended rationals and
variables are nonnegative rationals.


\section {Obtaining Farkas for equalities}

\begin{lstlisting}
theorem equalityFarkas (A : Matrix I J F) (b : I → F) :
    (∃ x : J → F, 0 ≤ x ∧ A *ᵥ x = b) ≠ (∃ y : I → F, 0 ≤ Aᵀ *ᵥ y ∧ b ⬝ᵥ y < 0)
\end{lstlisting}
We prove \texttt{equalityFarkas} as a corollary of the following theorem:
\begin{lstlisting}
theorem coordinateFarkas [Fintype J] [LinearOrderedDivisionRing R]
    (A : (I → R) →ₗ[R] J → R) (b : (I → R) →ₗ[R] R) :
    (∃ x : J → R, 0 ≤ x ∧ ∀ w : I → R, ∑ j : J, A w j • x j = b w) ≠ (∃ y : I → R, A y ≤ 0 ∧ 0 < b y)
\end{lstlisting}
We replaced the matrix \texttt{A} by a linear map
from a module over \texttt{R} to a finite-dimensional module over \texttt{R}.
We prove \texttt{coordinateFarkas} as a special case of the following theorem:
\begin{lstlisting}
theorem scalarFarkas [Fintype J] [LinearOrderedDivisionRing R] [AddCommGroup W] [Module R W]
    (A : W →ₗ[R] J → R) (b : W →ₗ[R] R) :
    (∃ x : J → R, 0 ≤ x ∧ ∀ w : W, ∑ j : J, A w j • x j = b w) ≠ (∃ y : W, A y ≤ 0 ∧ 0 < b y)
\end{lstlisting}
We replaced the finite-dimensional module \texttt{I $\rightarrow$ R} by a general module over \texttt{R}.
We prove \texttt{scalarFarkas} as a special case of the following theorem:
\begin{lstlisting}
theorem fintypeFarkasBartl [Fintype J] [LinearOrderedDivisionRing R]
    [LinearOrderedAddCommGroup V] [Module R V] [PosSMulMono R V] [AddCommGroup W] [Module R W]
    (A : W →ₗ[R] J → R) (b : W →ₗ[R] V) :
    (∃ x : J → V, 0 ≤ x ∧ ∀ w : W, ∑ j : J, A w j • x j = b w) ≠ (∃ y : W, A y ≤ 0 ∧ 0 < b y)
\end{lstlisting}
We replaced certain occurrences of \texttt{R} by a linear-ordered module over \texttt{R}
whose ordering satisfies \texttt{PosSMulMono}.


\section {Farkas-Bartl theorem}

\begin{lstlisting}
theorem finFarkasBartl {n : ℕ} [LinearOrderedDivisionRing R]
    [LinearOrderedAddCommGroup V] [Module R V] [PosSMulMono R V] [AddCommGroup W] [Module R W]
    (A : W →ₗ[R] Fin n → R) (b : W →ₗ[R] V) :
    (∃ x : Fin n → V, 0 ≤ x ∧ ∀ w : W, ∑ j : Fin n, A w j • x j = b w) ≠ (∃ y : W, A y ≤ 0 ∧ 0 < b y)
\end{lstlisting}
Let \texttt{n} be a natural number.
Let \texttt{R} be a linear-ordered division ring.
Let \texttt{V} be a linear-ordered module over \texttt{R}.
Let \texttt{W} be a module over \texttt{R}.
Let \texttt{A} be a linear (w.r.t.~\texttt{R}) map from \texttt{W} to $\texttt{R}^{\texttt{n}}$.
Let \texttt{b} be a linear (w.r.t.~\texttt{R}) map from \texttt{W} to \texttt{V}.
We claim that exactly one of the following is true;
(1) there is nonnegative \texttt{x} in $\texttt{V}^{\texttt{n}}$ such that \texttt{b} is equal to
the sum of elementwise mutliplications of \texttt{x} by scalars from \texttt{A}, or
(2) there exists \texttt{y} in \texttt{W} such that \texttt{A} applied to \texttt{y}
gives a nonpositive vector from \texttt{V} and \texttt{b} applied to \texttt{y} gives
a positive vector from \texttt{V}.

The part ``exists at most one'' is straightforward.
Now assume there is no \texttt{y} with desired properties, i.e.,
every \texttt{y} in \texttt{W} such that \texttt{A}~applied to \texttt{y}
gives a nonpositive vector from \texttt{V} also gives a nonpositive vector from \texttt{V}
when \texttt{b} is applied to \texttt{y}.
We prove the existence of~\texttt{x} with desired properties by induction on \texttt{n}
with universally-quantified \texttt{A} and \texttt{b}.

In the base case, observe that \texttt{A y $\le$ 0} is a tautology.
We choose \texttt{x} to be the zero vector. The part \texttt{0 $\le$ x} is satisfied.
It remains to show that for every \texttt{w} in \texttt{W}, the application \texttt{b w}
is zero (because we sum over an empty set on the right-hand side).
We do so by using the assumption twice; for \texttt{w} and for \texttt{(-w)}.

We express the induction step of the ``at least one'' direction as a lemma:
\begin{lstlisting}
lemma industepFarkasBartl {m : ℕ} [LinearOrderedDivisionRing R]
    [LinearOrderedAddCommGroup V] [Module R V] [PosSMulMono R V] [AddCommGroup W] [Module R W]
    (ih : ∀ A₀ : W →ₗ[R] Fin m → R, ∀ b₀ : W →ₗ[R] V,
      (∀ y₀ : W, A₀ y₀ ≤ 0 → b₀ y₀ ≤ 0) →
        (∃ x₀ : Fin m → V, 0 ≤ x₀ ∧ ∀ w₀ : W, ∑ i₀ : Fin m, A₀ w₀ i₀ • x₀ i₀ = b₀ w₀))
    {A : W →ₗ[R] Fin m.succ → R} {b : W →ₗ[R] V} (hAb : ∀ y : W, A y ≤ 0 → b y ≤ 0) :
    ∃ x : Fin m.succ → V, 0 ≤ x ∧ ∀ w : W, ∑ i : Fin m.succ, A w i • x i = b w
\end{lstlisting}
We distinguish two cases.

First, if the assumption reduced to the first \texttt{m} lines of \texttt{A} still holds,
we can simply take $\texttt{x}_0$ from the induction hypothesis and attach one more zero vector
to the end. Such \texttt{x} is nonnegative, and $\texttt{b w = b}_0\texttt{ w}_0$ provides
the desired identity.

Otherwise, let \texttt{y'} witness \texttt{0 < b y'} while mapping by the first \texttt{m} rows
of \texttt{A} onto \texttt{y'} gives everything nonnegative.
Let's rescale \texttt{y'} by $\texttt{(A y' j)}^{\texttt{-1}}$ and call the result \texttt{y}.
We make a series of observations:
\begin{lstlisting}
A y j = 1

∀ w : W, A (w - (A w j • y)) j = 0

∀ w : W, chop A (w - (A w j • y)) ≤ 0 → b (w - (A w j • y)) ≤ 0

∀ w : W, chop A w - chop A (A w j • y) ≤ 0 → b w - b (A w j • y) ≤ 0

∀ w : W, (chop A - (A · j • chop A y)) w ≤ 0 → (b - (A · j • b y)) w ≤ 0
\end{lstlisting}
Now we are ready to apply the induction hypothesis.
TODO.


\section {Going to inequalities and linear programming}

TODO.


\section {Going to extended rationals}
\renewcommand{\arraystretch}{1.25}

Let $p$ and $q$ be rational numbers.
Addition on external rationals is defined as follows:
\begin{center}
	\begin{tabular}{ c || c | c | c | }
		+ & $\bot$ & $q$ & $\top$  \\
		\hline\hline
		$\bot$ & $\bot$ & $\bot$ & $\bot$  \\ 
		\hline
		$p$ & $\bot$ & $p\!+\!q$ & $\top$  \\ 
		\hline
		$\top$ & $\bot$ & $\top$ & $\top$ \\ 
		\hline
	\end{tabular}
\end{center}
Let $p$ and $q$ be rational numbers.
Multiplication on extended rationals is defined as follows:
\begin{center}
	\begin{tabular}{ c || c | c | c | c | c | }
		$\cdot$ & $\bot$ & $q<0$ & $0$ & $q>0$ & $\top$  \\
		\hline\hline
		$\bot$ & $\top$ & $\top$ & $\bot$ & $\bot$ & $\bot$  \\ 
		\hline
		$p<0$ & $\top$ & $p \cdot q$ & $0$ & $p \cdot q$ & $\bot$  \\ 
		\hline
		$0$ & $\bot$ & $0$ & $0$ & $0$ & $0$  \\ 
		\hline
		$p>0$ & $\bot$ & $p \cdot q$ & $0$ & $p \cdot q$ & $\top$  \\ 
		\hline
		$\top$ & $\bot$ & $\bot$ & $0$ & $\top$ & $\top$ \\ 
		\hline
	\end{tabular}
\end{center}
However, we don't need the full multiplication table.
We will only use nonnegative rationals $\bullet$ extended rational:
\begin{center}
	\begin{tabular}{ c || c | c | c | c | c | }
		$\bullet$ & $\bot$ & $q<0$ & $0$ & $q>0$ & $\top$  \\
		\hline\hline
		$0$ & $\bot$ & $0$ & $0$ & $0$ & $0$  \\ 
		\hline
		$p>0$ & $\bot$ & $p \cdot q$ & $0$ & $p \cdot q$ & $\top$  \\ 
		\hline
	\end{tabular}
\end{center}
The table can be furthermore simplified to:
\begin{center}
\begin{tabular}{ c || c | c | c | }
	$\bullet$ & $\bot$ & $q$ & $\top$  \\
	\hline\hline
	$0$ & $\bot$ & $0$ & $0$  \\ 
	\hline
	$p>0$ & $\bot$ & $p \cdot q$ & $\top$  \\ 
	\hline
\end{tabular}
\end{center}
The last table will be used to define vector times vector and to define
matrix times vector.
TODO.

\end{document}
